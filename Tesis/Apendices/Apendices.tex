\addcontentsline{toc}{chapter}{Apéndices}
\chapter*{Apéndices}
\appendix
\chapter{Códigos de simulaciones númericas}
Para una consulta completa del código, se recomienda consultar el repositorio en github en la siguiente dirección electrónica.
https://github.com/skypsnet/Modelaci-nPython

\section{Códigos del caso homogéneo}
El código para la simulación 1 en materiales homogéneos es la siguiente:

\lstset{literate=
  {á}{{\'a}}1
  {é}{{\'e}}1
  {í}{{\'i}}1
  {ó}{{\'o}}1
  {ú}{{\'u}}1
  {Á}{{\'A}}1
  {É}{{\'E}}1
  {Í}{{\'I}}1
  {Ó}{{\'O}}1
  {Ú}{{\'U}}1
  {ñ}{{\~n}}1
  {ü}{{\"u}}1
  {Ü}{{\"U}}1
}

\begin{lstlisting}[frame=single]

#!/usr/bin/env python
#-*- coding: 850 -*-

from __future__ import print_function
from fenics import *
import numpy as np
import random as ra
import matplotlib.pyplot as plt
import sys
reload(sys)
sys.setdefaultencoding('utf-8')

# Se crea la malla donde se define el dominio

mesh= RectangleMesh(Point(0,0),Point(200,100),20,10)
V = FunctionSpace(mesh, 'P', 1)
# Se definen las condiciones de fronteras

def frontera_I(x,frontera):
 tol=1E-14    
 if frontera:
  if x[0]<=tol: 
   return True
  else:
   return False 
 else:
  return False  

\end{lstlisting}

\newpage

\begin{lstlisting}[frame=single]
def frontera_D(x,frontera):
 tol=1E-14    
 if frontera:
  if abs(x[0]-200)<=tol: 
   return True
  else:
   return False 
 else:
  return False  


F_D = DirichletBC(V, Constant(10), frontera_D)
F_I = DirichletBC(V, Constant(100), frontera_I)
bc = [F_D,F_I]

#Se define el problema variacional

u=TrialFunction(V)
v=TestFunction(V)
f=Constant(0)
a=dot(grad(u),grad(v))*dx
g=Constant(0)
L=f*v*dx-g*v*ds

# Se realiza el cálculo de la solución

u= Function(V)
solve(a==L,u,bc)
#tau=project(-grad(u))
#flujo=tau1.vector()[:] 


# Ploteo de la solución
plt.figure()
ax= plt.subplot(111)  
im=plot(u)
plt.colorbar(im) 
lo=plot(-grad(u))
plt.title('Flujo de agua subterránea')
plt.ylabel('Elevación [m]')
plt.xlabel('Distancia [m]')
plt.show()

\end{lstlisting}



Para la simulación en perfil, las condiciones de frontera cambian de la siguiente forma:

\lstset{language=python,breaklines=true, basicstyle=\footnotesize}
\begin{lstlisting}[frame=single]

def frontera_S(x,frontera):
 tol=1E-14    
 if frontera:
  if abs(x[1]-100)<=tol: 


\end{lstlisting}   
\begin{lstlisting}[frame=single]   
   return True
  else:
   return False
   
F_S = DirichletBC(V, Expression('100-0.45*x[0]',degree=1), frontera_S)
bc = [F_S]


\end{lstlisting}

Las condiciones de frontera para el caso de valle intermontano fue la siguiente:

\lstset{language=python,breaklines=true, basicstyle=\footnotesize}
\begin{lstlisting}[frame=single]

#!/usr/bin/env python
#-*- coding: 850 -*-

"""
Script que resuelve la ecuacion de flujo estacionario de Toth para condiciones complejas de valle intermontano
y un medio heterogeneo (Modelo de 2) a partir del metodo de elemento (Fenics)
Por: Ricardo Balam Chagoya Morales

Condiciones de la funcion que describe la geometria del nivel freático

Amplitud de la ecuación senoidal de pie de montaña = 1 m
Amplitud de la ecuación senoidal de intermontana = 10 m
longitud de onda de le ecuación senoidal = 0.01 m
Angulo de pendiente = 1.4711 rad 
"""


from __future__ import print_function
from fenics import *
import numpy as np
import random as ra
import sys
reload(sys)
sys.setdefaultencoding('utf-8')

# Refinación de la malla  
celx=20
cely=10
# Conductividad hidraulica central del primer y el segundo estrato
k1= 10
k2= 10
# Datos del modelo
lonx=200
lony=100
pasox=lonx/celx
pasoy=lony/cely

# Crea el mallado rectangular con intervalos de 20 metros
mesh = RectangleMesh(Point(0,0), Point(200,100), celx, cely)
V = FunctionSpace(mesh, 'P', 1)

\end{lstlisting}
\newpage

\begin{lstlisting}[frame=single]
# Se definen las condiciones de frontera superior
# Condicion de pie de montana
u_I = Expression('500-x[0]*tan(1.4711)+1*(sin((1000*x[0])/cos(1.4711))/cos(1.4711))', degree=1)

def frontera_I(x, dentro_frontera):
 tol = 1E-14
 if dentro_frontera:
   if abs(x[1]-100)<=tol and abs(x[0])<=40:
    return True
   else:
    return False
 else:
  return False

# Condicion de valle intermontano 
u_C = Expression('100+10*(sin(1000*x[0]))', degree=1)

def frontera_C(x, dentro_frontera):
 tol = 1E-14
 if dentro_frontera:
   if abs(x[1]-100)<=tol and 40<=abs(x[0])<=160:
    return True
   else:
    return False
 else:
  return False

# COndicion de pie de montana derecho
u_D = Expression('-1500+x[0]*tan(1.4711)+1*(sin((1000*x[0])/cos(1.4711))/cos(1.4711))', degree=1)

def frontera_D(x, dentro_frontera):
 tol = 1E-14
 if dentro_frontera:
   if abs(x[1]-100)<=tol and abs(x[0])>=160:
    return True
   else:
    return False
 else:
  return False


CondIzquierda = DirichletBC(V, u_I, frontera_I)
CondCentral = DirichletBC(V, u_C, frontera_C)
CondDerecha = DirichletBC(V, u_D, frontera_D)

bc = [CondIzquierda,CondCentral,CondDerecha]

\end{lstlisting}

\newpage


\section{Códigos del caso heterogéneo simple}

Las secciones del código que definen la heterogeneidad para la vista en planta es la siguiente (se intercambian los valores de la conductividad para definir el tipo de estrato):

\lstset{language=python}
\begin{lstlisting}[frame=single]
class K(Expression):
 def set_k_values(self, k_0, k_1):
     self.k_0, self.k_1 = k_0, k_1
 def eval(self, value, x):
   tol = 1E-14 
   if  x[0] <= 100 + tol:
      value[0] = self.k_0
   else:
      value[0] = self.k_1

kappa = K(degree=1)
kappa.set_k_values(10,100)

\end{lstlisting}

Para el caso de la vista en perfil:

\lstset{language=python,breaklines=true, basicstyle=\footnotesize}
\begin{lstlisting}[frame=single]
class K(Expression):
 def set_k_values(self, k_0, k_1):
     self.k_0, self.k_1 = k_0, k_1
 def eval(self, value, x):
   tol = 1E-14 
   if  x[1] <= 200 + tol:
      value[0] = self.k_0
   else:
      value[0] = self.k_1

kappa = K(degree=1)
kappa.set_k_values(10,100)

\end{lstlisting}

\section{Códigos del caso heterogéneo aleatorio}

La sección que define la heterogeneidad aleatoria es la siguiente:

\begin{lstlisting}[frame=single]
# Se define la heterogeneidad en el sistema

# Se realiza el marcado de la malla
k=range(0,400,1)
subdomains=CellFunction('size_t',mesh,0)
cont=0
for cell in cells(mesh):
 subdomains[cell]=k[cont]
 cont=cont+1
\end{lstlisting}

\begin{lstlisting}[frame=single]
plt.figure()
im1=plot(subdomains)
print(subdomains)
plt.colorbar(im1)
\end{lstlisting}


\begin{lstlisting}[frame=single]
# Se crea un espacio de funciones que simbolizan la conductividad hidráulica   
V0= FunctionSpace(mesh,"DG",0)
k=Function(V0)

# Se asigna para cada valor del marcado un valor del conjunto de conductividades
ka = np.loadtxt("Pruebayy.txt",delimiter=',',skiprows=1,usecols=[6])
k_values=np.exp(ka)
cont=0
for cell_no in cells(mesh):
 subdomain_no=subdomains[cell_no]
 k.vector()[cont]=k_values[subdomain_no]
 cont=cont+1


\end{lstlisting}

\chapter{Códigos en R para implementación de heterogeneidad}
\section{Busqueda de centroides y simulación no condicional}

\begin{lstlisting}[frame=single]
# Script en R que a partir de la biblioteca gstat calcula los centroides de cada elemento finito y crea el conjunto de valores de simulacion no condicional y genera un conjunto de valores de carga hidraulica a partir de un variograma exponencial

library("gstat")
xy <- expand.grid(1:5, 1:5)

################ Busqueda de centroides ###########################
b=10
h=10
x=matrix(0,1,40)
y=matrix(0,1,20)
yy=matrix(0,1,400)
# Ciclo que obtiene la coordenada x de los centroides
contx=1
for(i in 1:20){
 for(j in 1:2){
  if (contx%%2==0) 
 { 
  x[contx]<-((2*b)/3)+b*(i-1)
 }
  else 
 {
  x[contx]<-(b/3)+b*(i-1)
 }
 contx=contx+1
}
}
\end{lstlisting}
\newpage

\begin{lstlisting}[frame=single]

# Ciclo que obtiene la coordenada y de los centroides
conty=1
for(m in 1:10){
 for(n in 1:2){
  if (conty%%2==0) 
 { 
  y[conty]<-((h)/3)+h*(m-1)
 }
  else 
 {
  y[conty]<-(2*h/3)+h*(m-1)
 }
 conty=conty+1
}
}

# Se crea el vector de coordenadas x por elemento
xx=c(x,x,x,x,x,x,x,x,x,x)
# Se crea el vector de coordenadas y por elemento
contyy=1
aux=1
for(i in 1:10){
 for(j in 1:40){
  if(contyy%%2==0){
   yy[contyy]<-y[aux+1]}
  else{
   yy[contyy]<-y[aux]}
  contyy=contyy+1}
 aux=aux+2}
yy=as.vector(yy)
# Se crea un data frame para los centroides
xy=data.frame(xx,yy)
names(xy) <- c("x","y")

##################Simulacion no condicional############################

# Se realizan las simulaciones no condicionales
vgm1=vgm(1,"Exp",15)
g.dummy <- gstat(formula = z~1, locations = ~x+y, dummy = TRUE, beta = 0,model = vgm1, nmax = 20)
yy <- predict(g.dummy, newdata = xy, nsim = 4)
write.csv(yy, file="Pruebayy.csv")


\end{lstlisting}

