\chapter*{Resumen}
\addcontentsline{toc}{chapter}{Resumen}

Esta tesis, presenta los resultados de simulaciones numéricas de diferentes escenarios sintéticos de flujo de agua subterránea, haciendo principal énfasis en la implementación de la plataforma computacional FeniCS escrito en el lenguaje de programación Python, para resolver escenarios donde la conductividad hidráulica varia dentro del dominio.  
\\

La plataforma computacional FeniCS hace uso del método de elementos finitos para la resolución de la ecuación de flujo, consiguiendo resolver exitosamente casos de flujo en materiales homogéneos y con una heterogeneidad predefinida, además, debido a la naturaleza del FEM, permite el uso de geometrías irregulares e imposición de condiciones de frontera complejas, que representa una gran parte de las situaciones encontradas en la práctica. Además, a partir del acoplamiento con el paquete Gstat del lenguaje de programación R, permite generar simulaciones no condicionales que recrean la naturaleza aleatoria de la conductividad hidráulica en un medio poroso, siendo una herramienta de código abierto con gran potencial para el estudio de flujo de agua subterránea.


